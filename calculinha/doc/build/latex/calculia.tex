%% Generated by Sphinx.
\def\sphinxdocclass{report}
\documentclass[letterpaper,10pt,spanish]{sphinxmanual}
\ifdefined\pdfpxdimen
   \let\sphinxpxdimen\pdfpxdimen\else\newdimen\sphinxpxdimen
\fi \sphinxpxdimen=.75bp\relax

\PassOptionsToPackage{warn}{textcomp}
\usepackage[utf8]{inputenc}
\ifdefined\DeclareUnicodeCharacter
% support both utf8 and utf8x syntaxes
  \ifdefined\DeclareUnicodeCharacterAsOptional
    \def\sphinxDUC#1{\DeclareUnicodeCharacter{"#1}}
  \else
    \let\sphinxDUC\DeclareUnicodeCharacter
  \fi
  \sphinxDUC{00A0}{\nobreakspace}
  \sphinxDUC{2500}{\sphinxunichar{2500}}
  \sphinxDUC{2502}{\sphinxunichar{2502}}
  \sphinxDUC{2514}{\sphinxunichar{2514}}
  \sphinxDUC{251C}{\sphinxunichar{251C}}
  \sphinxDUC{2572}{\textbackslash}
\fi
\usepackage{cmap}
\usepackage[T1]{fontenc}
\usepackage{amsmath,amssymb,amstext}
\usepackage{babel}



\usepackage{times}
\expandafter\ifx\csname T@LGR\endcsname\relax
\else
% LGR was declared as font encoding
  \substitutefont{LGR}{\rmdefault}{cmr}
  \substitutefont{LGR}{\sfdefault}{cmss}
  \substitutefont{LGR}{\ttdefault}{cmtt}
\fi
\expandafter\ifx\csname T@X2\endcsname\relax
  \expandafter\ifx\csname T@T2A\endcsname\relax
  \else
  % T2A was declared as font encoding
    \substitutefont{T2A}{\rmdefault}{cmr}
    \substitutefont{T2A}{\sfdefault}{cmss}
    \substitutefont{T2A}{\ttdefault}{cmtt}
  \fi
\else
% X2 was declared as font encoding
  \substitutefont{X2}{\rmdefault}{cmr}
  \substitutefont{X2}{\sfdefault}{cmss}
  \substitutefont{X2}{\ttdefault}{cmtt}
\fi


\usepackage[Sonny]{fncychap}
\ChNameVar{\Large\normalfont\sffamily}
\ChTitleVar{\Large\normalfont\sffamily}
\usepackage{sphinx}

\fvset{fontsize=\small}
\usepackage{geometry}


% Include hyperref last.
\usepackage{hyperref}
% Fix anchor placement for figures with captions.
\usepackage{hypcap}% it must be loaded after hyperref.
% Set up styles of URL: it should be placed after hyperref.
\urlstyle{same}

\addto\captionsspanish{\renewcommand{\contentsname}{Contidos:}}

\usepackage{sphinxmessages}
\setcounter{tocdepth}{1}



\title{Calculiña}
\date{04 de mayo de 2021}
\release{1.0.0}
\author{Miguel López Patricio}
\newcommand{\sphinxlogo}{\vbox{}}
\renewcommand{\releasename}{Versión}
\makeindex
\begin{document}

\ifdefined\shorthandoff
  \ifnum\catcode`\=\string=\active\shorthandoff{=}\fi
  \ifnum\catcode`\"=\active\shorthandoff{"}\fi
\fi

\pagestyle{empty}
\sphinxmaketitle
\pagestyle{plain}
\sphinxtableofcontents
\pagestyle{normal}
\phantomsection\label{\detokenize{index::doc}}


\sphinxAtStartPar
Calculiña é unha aplicación para que os nenos e nenas comencen a xogar cos números
e a realizar as súas primeiras operacións matemáticas.
Foi realizada pensando nos rapaces e por iso se lle engadiron divertidas e coloridas iconas aos botóns.
En futuras versións se lle engadirán sons e xogos relacionados coas matemáticas.

\sphinxAtStartPar
A aplicación foi realizada en Python usando a libraría gráfica PySide2.

\noindent\sphinxincludegraphics[width=350\sphinxpxdimen]{{calculinha}.jpg}


\chapter{Calculiña}
\label{\detokenize{README:calculina}}\label{\detokenize{README::doc}}
\begin{figure}[htbp]
\centering

\noindent\sphinxincludegraphics{{calculinha}.jpg}
\end{figure}


\section{Recursos necesarios}
\label{\detokenize{README:recursos-necesarios}}\begin{itemize}
\item {} 
\sphinxAtStartPar
Documentación oficial Python:
\sphinxurl{https://docs.python.org/es/3/}

\item {} 
\sphinxAtStartPar
PySide2 (Qt 5 sobre Python):
\sphinxurl{https://doc.qt.io/qtforpython-5/contents.html}

\item {} 
\sphinxAtStartPar
Documentación de proxectos Python con Sphinx:
\sphinxurl{https://www.sphinx-doc.org/en/master/}

\item {} 
\sphinxAtStartPar
Guía para a sintaxe markdown:
\sphinxurl{https://www.markdownguide.org/basic-syntax/}

\end{itemize}


\section{Instalación das dependencias}
\label{\detokenize{README:instalacion-das-dependencias}}
\sphinxAtStartPar
En Linux xa temos o intérprete de Python no sistema.

\sphinxAtStartPar
Pero en Windows é necesaria a instalación.
\begin{itemize}
\item {} 
\sphinxAtStartPar
Descargar \sphinxurl{https://www.python.org/ftp/python/3.9.4/python-3.9.4-amd64.exe}

\end{itemize}
\begin{quote}

\sphinxAtStartPar
e instalar como administrador.
\end{quote}

\sphinxAtStartPar
Instalar tamén a seguinte dependencia en Linux:

\sphinxAtStartPar
\sphinxcode{\sphinxupquote{sudo apt\sphinxhyphen{}get install python3 python3\sphinxhyphen{}pip}}

\sphinxAtStartPar
En Windows seguir as instruccións:

\sphinxAtStartPar
\sphinxurl{https://recursospython.com/guias-y-manuales/instalacion-y-utilizacion-de-pip-en-windows-linux-y-os-x/}


\section{Execución da aplicación}
\label{\detokenize{README:execucion-da-aplicacion}}
\sphinxAtStartPar
\sphinxcode{\sphinxupquote{python calculadora\_infantil.py}}


\section{Xerar a documentación}
\label{\detokenize{README:xerar-a-documentacion}}\begin{enumerate}
\sphinxsetlistlabels{\arabic}{enumi}{enumii}{}{.}%
\item {} 
\sphinxAtStartPar
Inicialmente haberá que instalar o paquete correspondente:
\sphinxcode{\sphinxupquote{pip install zerovm\sphinxhyphen{}sphinx\sphinxhyphen{}theme}}

\item {} 
\sphinxAtStartPar
Crear unha carpeta doc e situarse nela e executar \sphinxcode{\sphinxupquote{sphinx\sphinxhyphen{}quickstart}}

\item {} 
\sphinxAtStartPar
Modificar/engadir as seguintes liñas en /doc/conf.py:

\sphinxAtStartPar
\sphinxcode{\sphinxupquote{import zerovm\_sphinx\_theme}}

\sphinxAtStartPar
\sphinxcode{\sphinxupquote{html\_theme = \textquotesingle{}zerovm\textquotesingle{}}}

\sphinxAtStartPar
\sphinxcode{\sphinxupquote{html\_theme\_path = {[}zerovm\_sphinx\_theme.theme\_path{]}}}

\item {} 
\sphinxAtStartPar
Engadir as extensións en /doc/conf.py:
\begin{quote}

\sphinxAtStartPar
{[} «sphinx.ext.intersphinx»,

\sphinxAtStartPar
«sphinx.ext.autodoc»,

\sphinxAtStartPar
«sphinx.ext.mathjax»,

\sphinxAtStartPar
«sphinx.ext.viewcode»,{]}
\end{quote}

\item {} 
\sphinxAtStartPar
Xerar os documentos REST:
\sphinxcode{\sphinxupquote{sphinx\sphinxhyphen{}apidoc \sphinxhyphen{}f \sphinxhyphen{}M \sphinxhyphen{}o source/api/ ../..}}

\item {} 
\sphinxAtStartPar
Xerar \sphinxcode{\sphinxupquote{README.rst}} e \sphinxcode{\sphinxupquote{index.rst}}

\item {} 
\sphinxAtStartPar
Xerar a documentación (en Windows usar \sphinxcode{\sphinxupquote{.\textbackslash{}make}}):
HTML: \sphinxcode{\sphinxupquote{make html}}
PDF: \sphinxcode{\sphinxupquote{make latexpdf}}

\end{enumerate}

\sphinxAtStartPar
Os resultados atoparanse en:
\begin{itemize}
\item {} 
\sphinxAtStartPar
\sphinxcode{\sphinxupquote{/doc/build/html}} e en \sphinxcode{\sphinxupquote{/doc/build/latexpdf}}

\end{itemize}

\sphinxAtStartPar
Se fose necesario desfacer o realizado executar:
\begin{itemize}
\item {} 
\sphinxAtStartPar
Terminal Linux: \sphinxcode{\sphinxupquote{make clean}}

\item {} 
\sphinxAtStartPar
Consola Windows: \sphinxcode{\sphinxupquote{.\textbackslash{}make clean}}

\end{itemize}


\chapter{modulos package}
\label{\detokenize{api/modulos:module-modulos}}\label{\detokenize{api/modulos:modulos-package}}\label{\detokenize{api/modulos::doc}}\index{módulo@\spxentry{módulo}!modulos@\spxentry{modulos}}\index{modulos@\spxentry{modulos}!módulo@\spxentry{módulo}}

\section{calculadora\_infantil module}
\label{\detokenize{api/modulos:module-calculadora_infantil}}\label{\detokenize{api/modulos:calculadora-infantil-module}}\index{módulo@\spxentry{módulo}!calculadora\_infantil@\spxentry{calculadora\_infantil}}\index{calculadora\_infantil@\spxentry{calculadora\_infantil}!módulo@\spxentry{módulo}}
\sphinxAtStartPar
Módulo principal da aplicación Calculiña
\index{CalculinhaApp (clase en calculadora\_infantil)@\spxentry{CalculinhaApp}\spxextra{clase en calculadora\_infantil}}

\begin{fulllineitems}
\phantomsection\label{\detokenize{api/modulos:calculadora_infantil.CalculinhaApp}}\pysigline{\sphinxbfcode{\sphinxupquote{class }}\sphinxcode{\sphinxupquote{calculadora\_infantil.}}\sphinxbfcode{\sphinxupquote{CalculinhaApp}}}
\sphinxAtStartPar
Bases: \sphinxcode{\sphinxupquote{PySide2.QtWidgets.QMainWindow}}

\sphinxAtStartPar
Clase CalculinhaApp do módulo principal da aplicación
Xera a ventana principal que amosará a calculadora
\index{closeEvent() (método de calculadora\_infantil.CalculinhaApp)@\spxentry{closeEvent()}\spxextra{método de calculadora\_infantil.CalculinhaApp}}

\begin{fulllineitems}
\phantomsection\label{\detokenize{api/modulos:calculadora_infantil.CalculinhaApp.closeEvent}}\pysiglinewithargsret{\sphinxbfcode{\sphinxupquote{closeEvent}}}{\emph{\DUrole{n}{evento}}}{}
\sphinxAtStartPar
Método para preguntar ao usuario se confirma a saída da aplicación
\begin{quote}\begin{description}
\item[{Parámetros}] \leavevmode
\sphinxAtStartPar
\sphinxstyleliteralstrong{\sphinxupquote{param2}} (\sphinxstyleliteralemphasis{\sphinxupquote{event}}) \textendash{} evento co que se relaciona.

\end{description}\end{quote}

\end{fulllineitems}

\index{escribindo\_segundo\_num (atributo de calculadora\_infantil.CalculinhaApp)@\spxentry{escribindo\_segundo\_num}\spxextra{atributo de calculadora\_infantil.CalculinhaApp}}

\begin{fulllineitems}
\phantomsection\label{\detokenize{api/modulos:calculadora_infantil.CalculinhaApp.escribindo_segundo_num}}\pysigline{\sphinxbfcode{\sphinxupquote{escribindo\_segundo\_num}}\sphinxbfcode{\sphinxupquote{ = False}}}
\sphinxAtStartPar
Variable escribindo\_segundo\_num
\begin{quote}\begin{description}
\item[{Type}] \leavevmode
\sphinxAtStartPar
boolean

\end{description}\end{quote}

\end{fulllineitems}

\index{limpa\_display() (método de calculadora\_infantil.CalculinhaApp)@\spxentry{limpa\_display()}\spxextra{método de calculadora\_infantil.CalculinhaApp}}

\begin{fulllineitems}
\phantomsection\label{\detokenize{api/modulos:calculadora_infantil.CalculinhaApp.limpa_display}}\pysiglinewithargsret{\sphinxbfcode{\sphinxupquote{limpa\_display}}}{}{}
\sphinxAtStartPar
Método que borra a pantalla amosando un cero

\end{fulllineitems}

\index{manual\_app() (método de calculadora\_infantil.CalculinhaApp)@\spxentry{manual\_app()}\spxextra{método de calculadora\_infantil.CalculinhaApp}}

\begin{fulllineitems}
\phantomsection\label{\detokenize{api/modulos:calculadora_infantil.CalculinhaApp.manual_app}}\pysiglinewithargsret{\sphinxbfcode{\sphinxupquote{manual\_app}}}{}{}
\sphinxAtStartPar
Método que amosará o manual do código da aplicación en formato pdf
Crea un obxecto da clase PdfWindow e chama ao método showPdf
enviando a ruta e nome do arquivo pdf

\end{fulllineitems}

\index{preme\_num() (método de calculadora\_infantil.CalculinhaApp)@\spxentry{preme\_num()}\spxextra{método de calculadora\_infantil.CalculinhaApp}}

\begin{fulllineitems}
\phantomsection\label{\detokenize{api/modulos:calculadora_infantil.CalculinhaApp.preme_num}}\pysiglinewithargsret{\sphinxbfcode{\sphinxupquote{preme\_num}}}{}{}
\sphinxAtStartPar
Método que amosará os números que se premen

\end{fulllineitems}

\index{preme\_operacion() (método de calculadora\_infantil.CalculinhaApp)@\spxentry{preme\_operacion()}\spxextra{método de calculadora\_infantil.CalculinhaApp}}

\begin{fulllineitems}
\phantomsection\label{\detokenize{api/modulos:calculadora_infantil.CalculinhaApp.preme_operacion}}\pysiglinewithargsret{\sphinxbfcode{\sphinxupquote{preme\_operacion}}}{}{}
\sphinxAtStartPar
Cando o usuario preme algún botón de operación
este método gardará o número amosado na pantalla como primeiro número

\end{fulllineitems}

\index{preme\_resultado() (método de calculadora\_infantil.CalculinhaApp)@\spxentry{preme\_resultado()}\spxextra{método de calculadora\_infantil.CalculinhaApp}}

\begin{fulllineitems}
\phantomsection\label{\detokenize{api/modulos:calculadora_infantil.CalculinhaApp.preme_resultado}}\pysiglinewithargsret{\sphinxbfcode{\sphinxupquote{preme\_resultado}}}{}{}
\sphinxAtStartPar
Método para ofrecer o resultado da operación
Cando o usuario preme “=” garda o número que se estea amosando
e se ofrecerá o resultado da operación
Pon a False o botón da operación realizada

\end{fulllineitems}

\index{primeiro\_num (atributo de calculadora\_infantil.CalculinhaApp)@\spxentry{primeiro\_num}\spxextra{atributo de calculadora\_infantil.CalculinhaApp}}

\begin{fulllineitems}
\phantomsection\label{\detokenize{api/modulos:calculadora_infantil.CalculinhaApp.primeiro_num}}\pysigline{\sphinxbfcode{\sphinxupquote{primeiro\_num}}\sphinxbfcode{\sphinxupquote{ = None}}}
\sphinxAtStartPar
Variable primeiro\_num
\begin{quote}\begin{description}
\item[{Type}] \leavevmode
\sphinxAtStartPar
float

\end{description}\end{quote}

\end{fulllineitems}

\index{sair() (método de calculadora\_infantil.CalculinhaApp)@\spxentry{sair()}\spxextra{método de calculadora\_infantil.CalculinhaApp}}

\begin{fulllineitems}
\phantomsection\label{\detokenize{api/modulos:calculadora_infantil.CalculinhaApp.sair}}\pysiglinewithargsret{\sphinxbfcode{\sphinxupquote{sair}}}{}{}
\sphinxAtStartPar
Método para sair da aplicación

\end{fulllineitems}

\index{staticMetaObject (atributo de calculadora\_infantil.CalculinhaApp)@\spxentry{staticMetaObject}\spxextra{atributo de calculadora\_infantil.CalculinhaApp}}

\begin{fulllineitems}
\phantomsection\label{\detokenize{api/modulos:calculadora_infantil.CalculinhaApp.staticMetaObject}}\pysigline{\sphinxbfcode{\sphinxupquote{staticMetaObject}}\sphinxbfcode{\sphinxupquote{ = \textless{}PySide2.QtCore.QMetaObject object\textgreater{}}}}
\end{fulllineitems}


\end{fulllineitems}



\section{modulos.InfoVentanaAp module}
\label{\detokenize{api/modulos:module-modulos.InfoVentanaAp}}\label{\detokenize{api/modulos:modulos-infoventanaap-module}}\index{módulo@\spxentry{módulo}!modulos.InfoVentanaAp@\spxentry{modulos.InfoVentanaAp}}\index{modulos.InfoVentanaAp@\spxentry{modulos.InfoVentanaAp}!módulo@\spxentry{módulo}}
\sphinxAtStartPar
Módulo da clase InfoVentana
para xerar a ventana de información da aplicación
\index{InfoVentana (clase en modulos.InfoVentanaAp)@\spxentry{InfoVentana}\spxextra{clase en modulos.InfoVentanaAp}}

\begin{fulllineitems}
\phantomsection\label{\detokenize{api/modulos:modulos.InfoVentanaAp.InfoVentana}}\pysigline{\sphinxbfcode{\sphinxupquote{class }}\sphinxcode{\sphinxupquote{modulos.InfoVentanaAp.}}\sphinxbfcode{\sphinxupquote{InfoVentana}}}
\sphinxAtStartPar
Bases: \sphinxcode{\sphinxupquote{PySide2.QtWidgets.QMainWindow}}

\sphinxAtStartPar
Clase InfoVentana que usaremos para crear unha ventana de información sobre a aplicación
\index{amosar\_ventana() (método de modulos.InfoVentanaAp.InfoVentana)@\spxentry{amosar\_ventana()}\spxextra{método de modulos.InfoVentanaAp.InfoVentana}}

\begin{fulllineitems}
\phantomsection\label{\detokenize{api/modulos:modulos.InfoVentanaAp.InfoVentana.amosar_ventana}}\pysiglinewithargsret{\sphinxbfcode{\sphinxupquote{amosar\_ventana}}}{}{}
\sphinxAtStartPar
Método para amosar ventana

\end{fulllineitems}

\index{staticMetaObject (atributo de modulos.InfoVentanaAp.InfoVentana)@\spxentry{staticMetaObject}\spxextra{atributo de modulos.InfoVentanaAp.InfoVentana}}

\begin{fulllineitems}
\phantomsection\label{\detokenize{api/modulos:modulos.InfoVentanaAp.InfoVentana.staticMetaObject}}\pysigline{\sphinxbfcode{\sphinxupquote{staticMetaObject}}\sphinxbfcode{\sphinxupquote{ = \textless{}PySide2.QtCore.QMetaObject object\textgreater{}}}}
\end{fulllineitems}


\end{fulllineitems}



\section{modulos.calculinhaWindow module}
\label{\detokenize{api/modulos:module-modulos.calculinhaWindow}}\label{\detokenize{api/modulos:modulos-calculinhawindow-module}}\index{módulo@\spxentry{módulo}!modulos.calculinhaWindow@\spxentry{modulos.calculinhaWindow}}\index{modulos.calculinhaWindow@\spxentry{modulos.calculinhaWindow}!módulo@\spxentry{módulo}}
\sphinxAtStartPar
Módulo calculinhaWindow
que servirá como modelo da calculadora
\index{Ui\_MainWindow (clase en modulos.calculinhaWindow)@\spxentry{Ui\_MainWindow}\spxextra{clase en modulos.calculinhaWindow}}

\begin{fulllineitems}
\phantomsection\label{\detokenize{api/modulos:modulos.calculinhaWindow.Ui_MainWindow}}\pysigline{\sphinxbfcode{\sphinxupquote{class }}\sphinxcode{\sphinxupquote{modulos.calculinhaWindow.}}\sphinxbfcode{\sphinxupquote{Ui\_MainWindow}}}
\sphinxAtStartPar
Bases: \sphinxcode{\sphinxupquote{object}}

\sphinxAtStartPar
Clase principal Ui\_MainWindow
Servirá como base da calculadora
\index{retranslateUi() (método de modulos.calculinhaWindow.Ui\_MainWindow)@\spxentry{retranslateUi()}\spxextra{método de modulos.calculinhaWindow.Ui\_MainWindow}}

\begin{fulllineitems}
\phantomsection\label{\detokenize{api/modulos:modulos.calculinhaWindow.Ui_MainWindow.retranslateUi}}\pysiglinewithargsret{\sphinxbfcode{\sphinxupquote{retranslateUi}}}{\emph{\DUrole{n}{MainWindow}}}{}
\end{fulllineitems}

\index{setupUi() (método de modulos.calculinhaWindow.Ui\_MainWindow)@\spxentry{setupUi()}\spxextra{método de modulos.calculinhaWindow.Ui\_MainWindow}}

\begin{fulllineitems}
\phantomsection\label{\detokenize{api/modulos:modulos.calculinhaWindow.Ui_MainWindow.setupUi}}\pysiglinewithargsret{\sphinxbfcode{\sphinxupquote{setupUi}}}{\emph{\DUrole{n}{MainWindow}}}{}
\end{fulllineitems}


\end{fulllineitems}



\section{modulos.info\_ventana module}
\label{\detokenize{api/modulos:module-modulos.info_ventana}}\label{\detokenize{api/modulos:modulos-info-ventana-module}}\index{módulo@\spxentry{módulo}!modulos.info\_ventana@\spxentry{modulos.info\_ventana}}\index{modulos.info\_ventana@\spxentry{modulos.info\_ventana}!módulo@\spxentry{módulo}}
\sphinxAtStartPar
Módulo da clase Ui\_infoWindow
Servirá para xerar unha ventana de info da app
\index{Ui\_infoWindow (clase en modulos.info\_ventana)@\spxentry{Ui\_infoWindow}\spxextra{clase en modulos.info\_ventana}}

\begin{fulllineitems}
\phantomsection\label{\detokenize{api/modulos:modulos.info_ventana.Ui_infoWindow}}\pysigline{\sphinxbfcode{\sphinxupquote{class }}\sphinxcode{\sphinxupquote{modulos.info\_ventana.}}\sphinxbfcode{\sphinxupquote{Ui\_infoWindow}}}
\sphinxAtStartPar
Bases: \sphinxcode{\sphinxupquote{object}}

\sphinxAtStartPar
Clase para xerar unha ventana de información da aplicación
\index{retranslateUi() (método de modulos.info\_ventana.Ui\_infoWindow)@\spxentry{retranslateUi()}\spxextra{método de modulos.info\_ventana.Ui\_infoWindow}}

\begin{fulllineitems}
\phantomsection\label{\detokenize{api/modulos:modulos.info_ventana.Ui_infoWindow.retranslateUi}}\pysiglinewithargsret{\sphinxbfcode{\sphinxupquote{retranslateUi}}}{\emph{\DUrole{n}{infoWindow}}}{}
\end{fulllineitems}

\index{setupUi() (método de modulos.info\_ventana.Ui\_infoWindow)@\spxentry{setupUi()}\spxextra{método de modulos.info\_ventana.Ui\_infoWindow}}

\begin{fulllineitems}
\phantomsection\label{\detokenize{api/modulos:modulos.info_ventana.Ui_infoWindow.setupUi}}\pysiglinewithargsret{\sphinxbfcode{\sphinxupquote{setupUi}}}{\emph{\DUrole{n}{infoWindow}}}{}
\end{fulllineitems}


\end{fulllineitems}



\section{modulos.novo\_boton module}
\label{\detokenize{api/modulos:module-modulos.novo_boton}}\label{\detokenize{api/modulos:modulos-novo-boton-module}}\index{módulo@\spxentry{módulo}!modulos.novo\_boton@\spxentry{modulos.novo\_boton}}\index{modulos.novo\_boton@\spxentry{modulos.novo\_boton}!módulo@\spxentry{módulo}}
\sphinxAtStartPar
Módulo novo\_boton
para xerar obxectos personalizados de tipo botón
\index{NovoBoton (clase en modulos.novo\_boton)@\spxentry{NovoBoton}\spxextra{clase en modulos.novo\_boton}}

\begin{fulllineitems}
\phantomsection\label{\detokenize{api/modulos:modulos.novo_boton.NovoBoton}}\pysiglinewithargsret{\sphinxbfcode{\sphinxupquote{class }}\sphinxcode{\sphinxupquote{modulos.novo\_boton.}}\sphinxbfcode{\sphinxupquote{NovoBoton}}}{\emph{\DUrole{n}{parent}\DUrole{o}{=}\DUrole{default_value}{None}}}{}
\sphinxAtStartPar
Bases: \sphinxcode{\sphinxupquote{PySide2.QtWidgets.QPushButton}}

\sphinxAtStartPar
Clase personalizada NovoBoton hereda de QPushButton
\index{engade\_icona\_info\_txt() (método de modulos.novo\_boton.NovoBoton)@\spxentry{engade\_icona\_info\_txt()}\spxextra{método de modulos.novo\_boton.NovoBoton}}

\begin{fulllineitems}
\phantomsection\label{\detokenize{api/modulos:modulos.novo_boton.NovoBoton.engade_icona_info_txt}}\pysiglinewithargsret{\sphinxbfcode{\sphinxupquote{engade\_icona\_info\_txt}}}{\emph{\DUrole{n}{icona}}, \emph{\DUrole{n}{txt}}}{}~\begin{description}
\item[{Método para engadir icona, axustar o seu tamaño e engadir axuda textual a cada obxecto da clase NovoBoton}] \leavevmode
\sphinxAtStartPar
Recibe como parámetros o nome da icona e un texto que formará parte da axuda textual do botón

\end{description}
\begin{quote}\begin{description}
\item[{Parámetros}] \leavevmode\begin{itemize}
\item {} 
\sphinxAtStartPar
\sphinxstyleliteralstrong{\sphinxupquote{param1}} (\sphinxstyleliteralemphasis{\sphinxupquote{str}}) \textendash{} nome do arquivo de imaxe a amosar no botón

\item {} 
\sphinxAtStartPar
\sphinxstyleliteralstrong{\sphinxupquote{param2}} (\sphinxstyleliteralemphasis{\sphinxupquote{str}}) \textendash{} axuda textual

\end{itemize}

\end{description}\end{quote}

\end{fulllineitems}

\index{staticMetaObject (atributo de modulos.novo\_boton.NovoBoton)@\spxentry{staticMetaObject}\spxextra{atributo de modulos.novo\_boton.NovoBoton}}

\begin{fulllineitems}
\phantomsection\label{\detokenize{api/modulos:modulos.novo_boton.NovoBoton.staticMetaObject}}\pysigline{\sphinxbfcode{\sphinxupquote{staticMetaObject}}\sphinxbfcode{\sphinxupquote{ = \textless{}PySide2.QtCore.QMetaObject object\textgreater{}}}}
\end{fulllineitems}


\end{fulllineitems}



\section{modulos.ventana\_pdf module}
\label{\detokenize{api/modulos:module-modulos.ventana_pdf}}\label{\detokenize{api/modulos:modulos-ventana-pdf-module}}\index{módulo@\spxentry{módulo}!modulos.ventana\_pdf@\spxentry{modulos.ventana\_pdf}}\index{modulos.ventana\_pdf@\spxentry{modulos.ventana\_pdf}!módulo@\spxentry{módulo}}
\sphinxAtStartPar
Módulo ventana\_pdf
para xerar a ventana que amosará o manual da aplicación
\index{PdfWindow (clase en modulos.ventana\_pdf)@\spxentry{PdfWindow}\spxextra{clase en modulos.ventana\_pdf}}

\begin{fulllineitems}
\phantomsection\label{\detokenize{api/modulos:modulos.ventana_pdf.PdfWindow}}\pysiglinewithargsret{\sphinxbfcode{\sphinxupquote{class }}\sphinxcode{\sphinxupquote{modulos.ventana\_pdf.}}\sphinxbfcode{\sphinxupquote{PdfWindow}}}{\emph{\DUrole{n}{parent}\DUrole{o}{=}\DUrole{default_value}{None}}}{}
\sphinxAtStartPar
Bases: \sphinxcode{\sphinxupquote{PySide2.QtWidgets.QMainWindow}}

\sphinxAtStartPar
Clase PdfWindow para xerar unha ventana que amosará o manual en pdf
\index{showPdf() (método de modulos.ventana\_pdf.PdfWindow)@\spxentry{showPdf()}\spxextra{método de modulos.ventana\_pdf.PdfWindow}}

\begin{fulllineitems}
\phantomsection\label{\detokenize{api/modulos:modulos.ventana_pdf.PdfWindow.showPdf}}\pysiglinewithargsret{\sphinxbfcode{\sphinxupquote{showPdf}}}{\emph{\DUrole{n}{dir\_arquivo}\DUrole{p}{:} \DUrole{n}{str}}}{}~\begin{description}
\item[{Método para amosar o pdf nunha ventana}] \leavevmode
\sphinxAtStartPar
Recibe como parámetro a ruta co nome do arquivo

\end{description}
\begin{quote}\begin{description}
\item[{Parámetros}] \leavevmode
\sphinxAtStartPar
\sphinxstyleliteralstrong{\sphinxupquote{dir\_arquivo}} (\sphinxstyleliteralemphasis{\sphinxupquote{str}}) \textendash{} Ruta ao pdf

\end{description}\end{quote}

\end{fulllineitems}

\index{staticMetaObject (atributo de modulos.ventana\_pdf.PdfWindow)@\spxentry{staticMetaObject}\spxextra{atributo de modulos.ventana\_pdf.PdfWindow}}

\begin{fulllineitems}
\phantomsection\label{\detokenize{api/modulos:modulos.ventana_pdf.PdfWindow.staticMetaObject}}\pysigline{\sphinxbfcode{\sphinxupquote{staticMetaObject}}\sphinxbfcode{\sphinxupquote{ = \textless{}PySide2.QtCore.QMetaObject object\textgreater{}}}}
\end{fulllineitems}


\end{fulllineitems}



\chapter{Indices e táboas}
\label{\detokenize{index:indices-e-taboas}}\begin{itemize}
\item {} 
\sphinxAtStartPar
\DUrole{xref,std,std-ref}{genindex}

\item {} 
\sphinxAtStartPar
\DUrole{xref,std,std-ref}{modindex}

\item {} 
\sphinxAtStartPar
\DUrole{xref,std,std-ref}{search}

\end{itemize}


\renewcommand{\indexname}{Índice de Módulos Python}
\begin{sphinxtheindex}
\let\bigletter\sphinxstyleindexlettergroup
\bigletter{c}
\item\relax\sphinxstyleindexentry{calculadora\_infantil}\sphinxstyleindexpageref{api/modulos:\detokenize{module-calculadora_infantil}}
\indexspace
\bigletter{m}
\item\relax\sphinxstyleindexentry{modulos}\sphinxstyleindexpageref{api/modulos:\detokenize{module-modulos}}
\item\relax\sphinxstyleindexentry{modulos.calculinhaWindow}\sphinxstyleindexpageref{api/modulos:\detokenize{module-modulos.calculinhaWindow}}
\item\relax\sphinxstyleindexentry{modulos.info\_ventana}\sphinxstyleindexpageref{api/modulos:\detokenize{module-modulos.info_ventana}}
\item\relax\sphinxstyleindexentry{modulos.InfoVentanaAp}\sphinxstyleindexpageref{api/modulos:\detokenize{module-modulos.InfoVentanaAp}}
\item\relax\sphinxstyleindexentry{modulos.novo\_boton}\sphinxstyleindexpageref{api/modulos:\detokenize{module-modulos.novo_boton}}
\item\relax\sphinxstyleindexentry{modulos.ventana\_pdf}\sphinxstyleindexpageref{api/modulos:\detokenize{module-modulos.ventana_pdf}}
\end{sphinxtheindex}

\renewcommand{\indexname}{Índice}
\printindex
\end{document}